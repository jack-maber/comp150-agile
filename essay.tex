% Please do not change the document class
\documentclass{scrartcl}

% Please do not change these packages
\usepackage[hidelinks]{hyperref}
\usepackage[none]{hyphenat}
\usepackage{setspace}
\doublespace

% You may add additional packages here
\usepackage{amsmath}

% Please include a clear, concise, and descriptive title
\title{Can the way that user stories are prioritised and allocated I.E. "MoSCoW method" have an impact on the length of a sprint in the video games industry?/ Are User Story distribution methods used in regular Agile Development practice useful in the Games Industry?}

% Please do not change the subtitle
\subtitle{COMP150 - Agile Development Practice}

% Please put your student number in the author field
\author{1606119}

\begin{document}

\maketitle

\abstract{My agile development question revolves around one specific part of the Agile Development process and that is "User Stories", and how I intend to answer the aforementioned question is to delve deeper into first, what a User Story is and how it is used by members of the development team in the development cycle to create a finished product, and then go further and look at how the user stories are created in the first place, and thus prioritised amongst the team members and on the Scrum Board, and as there are different methods of prioritising these user stories such as Walking Skeleton and the MoSCow method, I will compare and contrast these individual methods, and by using my research gained from my noted sources, I will be able to ascertain which is the most efficient for teams in a games development environment, and of the list are going to impact development time the least. And from these reflections on the different methods and the ways that they change the development teams behaviour and work pattern I will be able to draw a final conclusion on my question and whether further research is required to gain a better understanding of the impact it can have on the development time of a project in the video games industry.} 

\section{Introduction}

As the principles and philosophy of Agile Development is now common practice in both software and games development, the smaller parts of the overall "Agile" umbrella become more and more important to pick apart and evaluate to make sure that development goes as smoothly as possible and that "valuable software" and software is created "early and continuous" \cite{agilemanifesto}, so that projects are delivered to customers on time, which after all is the first point of the Agile Manifesto. And as User Stories are the way that work is presented to members, their creation and distribution must be closely monitored to ensure that the scope of the project is not under or over evaluated, and that members of the team are not over encumbered with work or tasks they can't carry out efficiently, which would extend the length of a sprint; and there are several methods of User Story distribution that I have discovered through my research that are generally used in Agile Development, and my research has found that the majority of them do help but aren't carried out in video games due to the different development practices and outside factors overriding their usefulness, but the main takeaway from my essay is how User Stories could be better distributed to aid the security of the length of sprints.



\section{General Use Of Agile In Games}

Through my research, I have found that there is quite a divide on my topic, and the usefulness of such distribution methods, with my first source outlining a relatively large method that according to the source is "really simple and easy to understand" \cite{popli2014prioritising}, which is always important when working under such tight time constraints and milestones that are found in the modern Games Industry, however they also state that the method is only suitable for the "prioritization of small and medium size project can be calculated efficiently" \cite{popli2014prioritising}, which makes it good for general software development and Indie Games, where a usually multi-skilled team is working on a project and thus can cover several disciplines, however it seems to me that it would be quite redundant for typical AAA Development as staff are normally very specialised and won't normally cross over to other disciplines in normal development practice. To further back this up, I again turn to Source 2 even goes as to say that "Prioritization of user-stories is a difficult task in agile environment because of its volatile nature." \cite{popli2014prioritising}, which is especially true of games development as there are many factors that can affect the length of a sprint, as mentioned by Al-azawi, Ayesh and Obaidy, such as "Feature Creep" \cite{al2014towards} which entails "new functionality is added during the development phase to increase project scope and change schedule time" \cite{al2014towards} and these type of events are what I would describe as volatile and such can affect the sprint length on a project, along with causing other issues such as "increased error, possible defects and increased chances of failure" \cite{al2014towards} which again in turn can also lead to increased sprint time., but then again, they only state that this is applicable to "Unmanaged" \cite{al2014towards} Feature Creep  

\section{Conclusion}

Write your conclusion here. The conclusion should do more than summarise the essay, making clear the contribution of the work and highlighting key points, limitations, and outstanding questions. It should not introduce any new content or information.

%[1] - Fabio Petrillo and Marcelo Pimenta. "Is agility out there?: agile practices in game development" presented at the 28th ACM International Conference on Design of Communication, Federal University of Rio Grande do Sul (UFRGS), Porto Alegre, Brazil, 2010, pages 9-15\par

%[2] - Elina M. I. Koivisto and Riku Suomela. "Using prototypes in early pervasive game development" presented at the 2007 ACM SIGGRAPH symposium on Video games, Nokia Research Center, Tampere Finland, 2007, pages 149-156\par

%[3] - Rula Al-azawi, Aladdin Ayesh and Mohaned Al. Obaidy "Towards Agent-based Agile approach for Game Development Methodology" presented at 2014 World Congress on Computer Applications and Information Systems (WCCAIS), El Mouradi Hammamet 5 Yasmine Hammamet, Hammamet, Tunisia, 2014\par

%[4] - Rashmi Popli, Naresh Chauhan, Hemant Sharma. "Prioritising user stories in agile environment" Issues and Challenges in Intelligent Computing Techniques (ICICT), 2014 International Conference on, 7-8 Feb 2014\par

%[5] - Vinod Kumar Chauhan. "How to reduce user story reopen count in Scrum development?" presented at Computing for Sustainable Global Development (INDIACom), 2015 2nd International Conference on, 11-13th March 2014\par

%[6] - Garm Lucassen, Fabiano Dalpiaz, Jan Martijn E.M. van der Werf, Sjaak Brinkkemper. "Forging high-quality User Stories: Towards a discipline for Agile Requirements" presented at Requirements Engineering Conference (RE), 2015 IEEE 23rd International, 24-28 Aug. 2015\par

%[7] - Rob Galanakis (2014, Feb. 19)"Agile Game Development is Hard". Gamasutra [Online]. Available: http://www.gamasutra.com/blogs/RobGalanakis/20140219/211185/Agile_Game_Development_is_Hard.php  [Accessed: Nov. 22, 2016] 


\bibliographystyle{ieeetran}
\bibliography{references}

\end{document}
